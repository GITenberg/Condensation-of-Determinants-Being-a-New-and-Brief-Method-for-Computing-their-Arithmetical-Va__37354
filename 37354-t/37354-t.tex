% %%%%%%%%%%%%%%%%%%%%%%%%%%%%%%%%%%%%%%%%%%%%%%%%%%%%%%%%%%%%%%%%%%%%%%% %
%                                                                         %
% The Project Gutenberg EBook of Condensation of Determinants, Being a New%
% and Brief Method for Computing their Arithmetical Values, by Lewis Carroll (AKA Charles Lutwidge Dodgson)
%                                                                         %
% This eBook is for the use of anyone anywhere at no cost and with        %
% almost no restrictions whatsoever.  You may copy it, give it away or    %
% re-use it under the terms of the Project Gutenberg License included     %
% with this eBook or online at www.gutenberg.org                          %
%                                                                         %
%                                                                         %
% Title: Condensation of Determinants, Being a New and Brief Method for Computing their Arithmetical Values
%                                                                         %
% Author: Lewis Carroll (AKA Charles Lutwidge Dodgson)                    %
%                                                                         %
% Release Date: September 8, 2011 [EBook #37354]                          %
%                                                                         %
% Language: English                                                       %
%                                                                         %
% Character set encoding: ISO-8859-1                                      %
%                                                                         %
% *** START OF THIS PROJECT GUTENBERG EBOOK CONDENSATION OF DETERMINANTS ***
%                                                                         %
% %%%%%%%%%%%%%%%%%%%%%%%%%%%%%%%%%%%%%%%%%%%%%%%%%%%%%%%%%%%%%%%%%%%%%%% %

\def\ebook{37354}
%%%%%%%%%%%%%%%%%%%%%%%%%%%%%%%%%%%%%%%%%%%%%%%%%%%%%%%%%%%%%%%%%%%%%%
%%                                                                  %%
%% Packages and substitutions:                                      %%
%%                                                                  %%
%% book:     Required.                                              %%
%% inputenc: Standard DP encoding. Required.                        %%
%%                                                                  %%
%% amsmath:  AMS mathematics enhancements. Required.                %%
%% amssymb:  Additional mathematical symbols. Required.             %%
%%                                                                  %%
%% alltt:    Fixed-width font environment. Required.                %%
%% array:    Enhanced tabular features. Required.                   %%
%%                                                                  %%
%% geometry: Enhanced page layout package. Required.                %%
%% fancyhdr: Enhanced page headers. Required.                       %%
%% hyperref: Hypertext embellishments for pdf output. Required.     %%
%% ifthen:   Logical conditionals. Required.                        %%
%%                                                                  %%
%%                                                                  %%
%% Producer's Comments:                                             %%
%%                                                                  %%
%%   Changes are noted in this file with \DPtypo{}{}, showing       %%
%%      original and replacement text side-by-side.                 %%
%%                                                                  %%
%% PDF pages: 20                                                    %%
%% PDF page size: 5.25 x 8" (non-standard)                          %%
%% PDF document info: filled in                                     %%
%%                                                                  %%
%%                                                                  %%
%% Command block:                                                   %%
%%                                                                  %%
%%     pdflatex x2                                                  %%
%%                                                                  %%
%%                                                                  %%
%%                                                                  %%
%% September 2011: pglatex.                                         %%
%%   Compile this project with:                                     %%
%%   pdflatex 37354-t.tex ..... TWO times                           %%
%%                                                                  %%
%%   pdfTeX, Version 3.1415926-1.40.10 (TeX Live 2009/Debian)       %%
%%                                                                  %%
%%%%%%%%%%%%%%%%%%%%%%%%%%%%%%%%%%%%%%%%%%%%%%%%%%%%%%%%%%%%%%%%%%%%%%
\listfiles

\documentclass[12pt]{book}[2007/10/19]
\usepackage[latin1]{inputenc}[2008/03/30]

\usepackage{amsmath}[2000/07/18] %% Displayed equations
\usepackage{amssymb}[2009/06/22] %% and additional symbols

\usepackage{alltt}[1997/06/16]   %% boilerplate, credits, license

\usepackage{array}[2008/09/09]   %% extended array/tabular features

\usepackage{ifthen}[2001/05/26]

% for running heads
\usepackage{fancyhdr}

\newcommand{\Margins}{hmarginratio=1:1}     % Symmetric margins
\newcommand{\HLinkColor}{blue}              % Hyperlink color
\newcommand{\PDFPageLayout}{SinglePage}
\newcommand{\TransNote}{Transcriber's Note}
\newcommand{\TransNoteText}{%
  Minor typographical corrections and presentational changes have
  been made without comment.
  \bigskip
}

\renewcommand{\cleardoublepage}{\clearpage}

 \raggedbottom
 \usepackage[body={5in,6.66in}, paperwidth=5.25in, paperheight=8in, \Margins,includeheadfoot]{geometry}[2010/09/12]

\providecommand{\ebook}{00000}    % Overridden during white-washing
\usepackage[pdftex,
  hyperfootnotes=false,
  pdftitle={The Project Gutenberg eBook \#\ebook: Condensation of Determinants, being a new and brief Method for computing their arithmetical values},
  pdfauthor={Charles L. Dodgson (Lewis Carrol)},
  pdfkeywords={Carlo Traverso, Anna Hall,
               Project Gutenberg Online Distributed Proofreading Team},
  pdfstartview=Fit,    % default value
  pdfstartpage=1,      % default value
  pdfpagemode=UseNone, % default value
  bookmarks=true,      % default value
  linktocpage=false,   % default value
  pdfpagelayout=\PDFPageLayout,
  pdfdisplaydoctitle,
  pdfpagelabels=true,
  bookmarksopen=true,
  bookmarksopenlevel=1,
  colorlinks=true,
  linkcolor=\HLinkColor]{hyperref}[2011/04/17]

%%%% Fixed-width environment to format PG boilerplate %%%%
\newenvironment{PGtext}{%
\begin{alltt}
\fontsize{8.1}{9}\ttfamily\selectfont}%
{\end{alltt}}

% Running heads
\newcommand{\FlushRunningHeads}{\clearpage\fancyhf{}\cleardoublepage}

\newcommand{\BookMark}[2]{\phantomsection\pdfbookmark[#1]{#2}{#2}}

%% Major document divisions %%
\newcommand{\PGBoilerPlate}{%
  \pagenumbering{Alph}
  \pagestyle{empty}
  \BookMark{0}{PG Boilerplate.}
}
\newcommand{\FrontMatter}{%
  \cleardoublepage
  \frontmatter
  \pagestyle{empty}
}

\newcommand{\MainMatter}{%
  \FlushRunningHeads
  \mainmatter
  \pagestyle{fancy}
  \fancyhf{}
  \setlength{\headheight}{15pt}
  \thispagestyle{plain}
  \fancyhead[C]{Rev.\ C. L. Dodgson \textit{on Condensation of Determinants}.}
  \renewcommand{\headrulewidth}{0pt}
  \fancyhead[R]{\thepage}
}

\newcommand{\BackMatter}{%
  \FlushRunningHeads
  \backmatter
  \BookMark{-1}{Back Matter.}
}

\newcommand{\DPtypo}[2]{#2}

\newcommand{\SquashArrays}{\setlength{\arraycolsep}{2pt}}

\newcommand{\tb}{\begin{center}\rule{1in}{0.5pt}\end{center}}

\DeclareInputText{183}{\ifmmode\cdot\else\textperiodcentered\fi}

\newcounter{RowCount}
% \Tall{3}: Create vertical space as tall as an array with 3 rows
\newcommand{\Tall}[1]{\setcounter{RowCount}{1}%
 \vphantom{\left|\begin{array}{r}
       \whiledo{\value{RowCount} < #1}
       {\stepcounter{RowCount}0\\}
     0
     \end{array}\right|}}

\hfuzz=0.5pt
%%%%%%%%%%%%%%%%%%%%%%%% START OF DOCUMENT %%%%%%%%%%%%%%%%%%%%%%%%%%

\begin{document}
\PGBoilerPlate
{\small
\begin{PGtext}
The Project Gutenberg EBook of Condensation of Determinants, Being a New
and Brief Method for Computing their Arithmetical Values, by Lewis Carroll (AKA Charles Lutwidge Dodgson)

This eBook is for the use of anyone anywhere at no cost and with
almost no restrictions whatsoever.  You may copy it, give it away or
re-use it under the terms of the Project Gutenberg License included
with this eBook or online at www.gutenberg.org


Title: Condensation of Determinants, Being a New and Brief Method for Computing their Arithmetical Values

Author: Lewis Carroll (AKA Charles Lutwidge Dodgson)

Release Date: September 8, 2011 [EBook #37354]

Language: English

Character set encoding: ISO-8859-1

*** START OF THIS PROJECT GUTENBERG EBOOK CONDENSATION OF DETERMINANTS ***
\end{PGtext}
\clearpage
\begin{PGtext}
Produced by Carlo Traverso, Anna Hall and the Online
Distributed Proofreading Team at http://www.pgdp.net (This
book was produced from scanned images of public domain
material from the Google Print project.)
\end{PGtext}}
\vfill

\BookMark{0}{Transcriber's Note}
\subsection*{\centering\normalfont\scshape
\normalsize\MakeLowercase{\TransNote}}

\TransNoteText
\clearpage

%%%%%%%%%%%%%%%%%%%%%%%%%%% FRONT MATTER %%%%%%%%%%%%%%%%%%%%%%%%%%

\FrontMatter

%% -----File: 000.png---
\begin{center}
{\LARGE PROCEEDINGS} \\
\vfill
{\scriptsize OF THE }\\
\vfill
{\LARGE ROYAL SOCIETY OF LONDON.}
\vfill
\textit{From January~11, 1866, to May~23, 1867, inclusive.}
\vfill
VOL.~XV.
\vfill
LONDON: \\
PRINTED BY TAYLOR AND FRANCIS, \\
{\scriptsize RED LION COURT, FLEET STREET.} \\
{\footnotesize MDCCCLXVII.}
\end{center}
%% -----File: 001.png---
\MainMatter
\begin{quote}
IV. ``Condensation of Determinants, being a new and brief Method
for computing their arithmetical values.'' By the Rev.\ C.~L.
\textsc{Dodgson},~M.A., Student of Christ Church, Oxford. Communicated
by the Rev.\ \textsc{Bartholomew Price}, M.A., F.R.S\@.
Received May~15,~1866.
\end{quote}

If it be proposed to solve a set of $n$~simultaneous linear equations, not
being all homogeneous, involving $n$~unknowns, or to test their compatibility
when all are homogeneous, by the method of determinants, in these, as
well as in other cases of common occurrence, it is necessary to compute
the arithmetical values of one or more determinants---such, for example, as
\[
\left|\begin{array}{rrr}
1, &3, &-2\\
2, &1, & 4\\
3, &5, &-1
\end{array}\right|.
\]

Now the only method, so far as I am aware, that has been hitherto
employed for such a purpose, is that of multiplying each term of the first
row or column by the determinant of its complemental minor, and affecting
the products with the signs $+$~and~$-$ alternately, the determinants required
in the process being, in their turn, broken up in the same manner
until determinants are finally arrived at sufficiently small for mental computation.

This process, in the above instance, would run thus:---
\[
\begin{split}
\left|\begin{array}{rrr}
1, &3, &-2\\
2, &1, & 4\\
3, &5, &-1
\end{array}\right|
= 1 � \left|\begin{array}{rr}
1, & 4\\
5, &-1
\end{array}\right|
-2 � \left|\begin{array}{rr}
3, &-2\\
5, &-1
\end{array}\right|
+ 3 � \left|\begin{array}{rr}
3, &-2\\
1, & 4
\end{array}\right|\\
= -21 - 14 + 42 = 7.
\end{split}
\]

But such a process, when the block consists of $16$, $25$,~or more terms, is
so tedious that the old method of elimination is much to be preferred for
solving simultaneous equations; so that the new method, excepting for
equations containing $2$~or $3$~unknowns, is practically useless.

The new method of computation, which I now proceed to explain, and
for which ``Condensation'' appears to be an appropriate name, will be
found, I believe, to be far shorter and simpler than any hitherto employed.

In the following remarks I shall use the word ``Block'' to denote any
number of terms arranged in rows and columns, and ``interior of a block''
to denote the block which remains when the first and last rows and columns
are erased.

The process of ``Condensation'' is exhibited in the following rules, in
which the given block is supposed to consist of $n$~rows and $n$~columns:---

(1) Arrange the given block, if necessary, so that no ciphers occur in its
interior. This may be done either by transposing rows or columns, or by
adding to certain rows the several terms of other rows multiplied by
certain multipliers.

(2) Compute the determinant of every minor consisting of four adjacent
%% -----File: 002.png---
terms. These values will constitute a second block, consisting of $(n - 1)$
rows and $(n - 1)$~columns.

(3) Condense this second block in the same manner, dividing each term,
when found, by the corresponding term in the interior of the first block.

(4) Repeat this process as often as may be necessary (observing that in
condensing any block of the series, the $r$th for example, the terms so found
must be divided by the corresponding terms in the interior of the $(r - 1)$th
block), until the block is condensed to a single term, which will be the
required value.

As an instance of the foregoing rules, let us take the block
\[
\left|\begin{array}{rrrr}
-2 &-1 &-1 &-4\\
-1 &-2 &-1 &-6\\
-1 &-1 & 2 & 4\\
 2 & 1 &-3 &-8
\end{array}\right|.
\]

By rule~(2) this is condensed into
$\left|\begin{array}{rrr}
 3 &-1 & 2\\
-1 &-5 & 8\\
 1 & 1 &-4
\end{array}\right|
$;
this, again, by
rule~(3), is condensed into
$\left|\begin{array}{rr}
 8 &-2\\
-4 & 6
\end{array}\right|$; and this, by rule~(4), into~$-8$,
which is the required value.

The simplest method of working this rule appears to be to arrange the
series of blocks one under another, as here exhibited; it will then be found
very easy to pick out the divisors required in rules (3)~and~(4).
\[
\left|\begin{array}{rrrr}
-2 &-1 &-1 &-4\\
-1 &-2 &-1 &-6\\
-1 &-1 & 2 & 4\\
 2 & 1 &-3 &-8
\end{array}\right|
\]
\[
\left|\begin{array}{rrr}
 3 &-1 &2\\
-1 &-5 &8\\
 1 & 1 &-4
\end{array}\right|
\]
\[
\left|\begin{array}{rr}
 8 &-2\\
-4 & 6
\end{array}\right|
\]
\[
-8.
\]

This process cannot be continued when ciphers occur in the interior of
any one of the blocks, since infinite values would be introduced by employing
them as divisors. When they occur in the given block itself, it
may be rearranged as has been already mentioned; but this cannot be done
when they occur in any one of the derived blocks; in such a case the
given block must be rearranged as circumstances require, and the operation
commenced anew.

The best way of doing this is as follows:---

Suppose a cipher to occur in the $h$th~row and $k$th~column of one of the
derived blocks (reckoning both row and column from the \emph{nearest} corner
of the block); find the term in the $h$th~row and $k$th~column of the given
%% -----File: 003.png---
block (reckoning from the corresponding corner), and transpose rows or
columns cyclically until it is left in an outside row or column. When the
necessary alterations have been made in the derived blocks, it will be found
that the cipher now occurs in an outside row or column, and therefore
need no longer be used as a divisor.

The advantage of \emph{cyclical} transposition is, that most of the terms in the
new blocks will have been computed already, and need only be copied; in
no case will it be necessary to compute more than \emph{one} new row or column
for each block of the series.

In the following instance it will be seen that in the first series of blocks
a cipher occurs in the interior of the third. We therefore abandon the
process at that point and begin again, rearranging the given block by
transferring the top row to the bottom; and the cipher, when it occurs, is
now found in an exterior row. It will be observed that in each block of
the new series, there is only \emph{one} new row to be computed; the other rows
are simply copied from the work already done.
\[
\begin{gathered}[t]
\left|\begin{array}{rrrrr}
2 &-1 & 2 & 1 &-3\\
1 & 2 & 1 &-1 & 2\\
1 &-1 &-2 &-1 &-1\\
2 & 1 &-1 &-2 &-1\\
1 &-2 &-1 &-1 & 2\\
\end{array}\right|
\\
\left|\begin{array}{rrrr}
 5 &-5 &-3 &-1\\
-3 &-3 &-3 & 3\\
 3 & 3 & 3 &-1\\
-5 &-3 &-1 &-5\\
\end{array}\right|
\\
\left|\begin{array}{rrr}
\DPtypo{-30}{-15} & 6 &\DPtypo{-12}{12}\\
  0 & 0 &  6\\
  6 &-6 &  8\\
\end{array}\right|
\end{gathered}
\qquad
\begin{gathered}[t]
\left|\begin{array}{rrrrr}
1 & 2 & 1 &-1 & 2\\
1 &-1 &-2 &-1 &-1\\
2 & 1 &-1 &-2 &-1\\
1 &-2 &-1 &-1 & 2\\
2 &-1 & 2 & 1 &-3\\
\end{array}\right|
\\
\left|\begin{array}{rrrr}
-3 &-3 &-3 & 3\\
 3 & 3 & 3 &-1\\
-5 &-3 &-1 &-5\\
 3 &-5 & 1 & 1\\
\end{array}\right|
\\
\left|\begin{array}{rrr}
  0 & 0 & 6\\
  6 &-6 & 8\\
-17 & 8 &-4\\
\end{array}\right|
\\
\left|\begin{array}{rr}
 0 &12\\
18 &40\\
\end{array}\right|
\\
36.
\end{gathered}
\]

The fact that, whenever ciphers occur in the interior of a derived block,
it is necessary to recommence the operation, may be thought a great
obstacle to the use of this method; but I believe it will be found in practice
that, even though this should occur several times in the course of one
operation, the whole amount of labour will still be much less than that involved
in the old process of computation.
\tb

I now proceed to give a proof of the validity of this process, deduced
from a well-known theorem in determinants; and in doing so, I shall use
the word ``adjugate'' in the following sense:---if there be a square block,
and if a new block be formed, such that each of its terms is the determinant
of the complemental minor of the corresponding term of the first
block, the second block is said to be \emph{adjugate} to the first.
%% -----File: 004.png---

The theorem referred to is the following:---

``If the determinant of a block $= R$, the determinant of any minor of
the $m$th~degree of the adjugate block is the product of~$R^{m - 1}$ and the
coefficient which, in~$R$, multiplies the determinant of the corresponding
minor.''

Let us first take a block of $9$~terms,
\[
\left|\begin{array}{ccc}
a_{1,1} &a_{1,2} &a_{1,3}\\
a_{2,1} &a_{2,2} &a_{2,3}\\
a_{3,1} &a_{3,2} &a_{3,3}
\end{array}\right|
= R;
\]
and let $\alpha_{1,1}$~represent the determinant of the complemental minor of~$a_{1,1}$,
and so on.

If we ``condense'' this, by the method already given, we get the block
$\left\{\begin{array}{cc}
\alpha_{3,3} &\alpha_{3,1}\\
\alpha_{1,3} &\alpha_{1,1}
\end{array}\right\},$ and, by the theorem above cited, the determinant of this,
viz.\
\[
\left|\begin{array}{cc}
\alpha_{3,3} &\alpha_{3,1}\\
\alpha_{1,3} &\alpha_{1,1}
\end{array}\right|
= R � a_{2,2}.
\]

Hence
\[
R = \frac{\left|\begin{array}{cc}
\alpha_{3,3} &\alpha_{3,1}\\
\alpha_{1,3} &\alpha_{1,1}
\end{array}\right|}{a_{2,2}},
\]
which proves the rule.

Secondly, let us take a block of $16$~terms:
\[
\left|\begin{array}{ccc}
a_{1,1}&\dots &a_{1,4}\\
\vdots &      &\vdots \\
a_{4,1}&\dots &a_{4,4}
\end{array}\right|
= R.
\]
If we ``condense'' this, we get a block of $9$~terms; let us denote it by
\[
\left\{\begin{array}{ccc}
b_{1,1} &\dots &b_{1,3}\\
\vdots  &      &\vdots \\
b_{3,1} &\dots &b_{3,3}
\end{array}\right\},
\text{ in which }
b_{1,1} =
\left|\begin{array}{cc}
a_{1,1} &a_{1,2}\\
a_{2,1} &a_{2,2}
\end{array}\right|, \text{ \&c.}
\]

If we ``condense'' this block again, we get a block of $4$~terms, each
of which, by the preceding paragraph, is the determinant of $9$~terms
of the original block; that is to say, we get the block
$\left\{\begin{array}{cc}
\alpha_{4,4} &\alpha_{4,1}\\
\alpha_{1,4} &\alpha_{1,1}
\end{array}\right\}$;
but, by the theorem already quoted,
$\left|\begin{array}{cc}
\alpha_{4,4} &\alpha_{4,1}\\
\alpha_{1,4} &\alpha_{1,1}
\end{array}\right| = R � b_{2,2}$; therefore
$R = \dfrac{\left|\begin{array}{cc}
\alpha_{4,4} &\alpha_{4,1}\\
\alpha_{1,4} &\alpha_{1,1}
\end{array}\right|}{b_{2,2}}$; that is, $R$~may be obtained by ``condensing'' the block
$\left\{\begin{array}{cc}
\alpha_{4,4} &\alpha_{4,1}\\
\alpha_{1,4} &\alpha_{1,1}
\end{array}\right\}$.

This proves the rule for a block of $16$~terms; and similar proofs might
be given for larger blocks.
\tb

I shall conclude by showing how this process may be applied to the
solution of simultaneous linear equations.
%% -----File: 005.png---

If we take a block consisting of $n$~rows and $(n + 1)$~columns, and ``condense''
it, we reduce it at last to $2$~terms, the first of which is the determinant
of the first $n$~columns, the other of the last $n$~columns.

Hence, if we take the $n$~simultaneous equations,
\begin{alignat*}{3}
a_{1,1}x_{1} +{}& a_{1,2}x_{2} + \cdots\,&\cdots + a_{1,n}x_{n} &+ a_{1,n + 1} = 0,\\
\multispan{4}{\dotfill} \\
a_{n,1}x_{1} +{}&\multispan{2}{\dotfill}&+ a_{n,n + 1} = 0;
\end{alignat*}
and if we condense the whole block of coefficients and constants, viz.\
\[
\left\{\begin{array}{ccc}
a_{1,1} &\dots &a_{1,n + 1}\\
\vdots  &      &\vdots \\
a_{n,1} &\dots &a_{n,n + 1}
\end{array}\right\},
\]
we reduce it at last to $2$~terms: let us denote them by $S$,~$T$, so that
\[
S = \left|\begin{array}{ccc}
a_{1,1} &\dots &a_{1,n}\\
\vdots  &      &\vdots \\
a_{n,1} &\dots &a_{n,n}
\end{array}\right|,
\text{ and }
T = \left|\begin{array}{ccc}
a_{\DPtypo{1,1}{1,2}} &\dots &a_{1,n+1}\\
\vdots  &      &\vdots \\
a_{n,2} &\dots &a_{n,n+1}
\end{array}\right|.
\]

Now we know that $x_{1} = (-)^{n} \dfrac{T}{S}$, which may be written in the form
$(-)^{n}S � x_{1} = T$.

Hence the $2$~terms obtained by the process of condensation may be
converted into an equation for~$x_{1}$, by multiplying the first of them by~$x_{1}$,
affected with $+$~or~$-$, according as $n$~is even or odd. The latter part of
the rule may be simply expressed thus:---``place the signs $+$~and~$-$
alternately over the several columns, beginning with the last, and the sign
which occurs over the column containing~$x_{1}$ is the sign with which $x_{1}$ is to
be affected.''

When the value of~$x_{1}$ has been thus found, it may be substituted in the
first $(n - 1)$~equations, and the same operation repeated on the new block,
which will now consist of $(n - 1)$~rows and $n$~columns. But in calculating
the second series of blocks, it will be found that most of the work has been
already done; in fact, of the $2$~determinants required in the new block, one
has been already computed correctly, and the other so nearly so that it
only requires the \emph{last} column in each of the derived blocks to be corrected.

In the \hyperlink{example}{example} given opposite, after writing $+$~and~$-$ alternately over
the columns, beginning with the last, we first condense the whole block, and
thus obtain the $2$~terms $36$~and~$-72$. Observing that the $x$-column has
the sign~$-$ placed over it, we multiply the~$36$ by~$-x$, and so form the
equation $-36x = -72$, which gives~$x = 2$.

Hence the $x$-terms in the first four equations become respectively
$2$,~$2$,~$4$, and~$2$; adding these values to the constant terms in the same equations,
we obtain a block of which we need only write down the last two
columns, viz.\
\[
\left|\begin{array}{rr}
 2 & 4\\
-1 &-2\\
-1 &-2\\
 2 & 6
\end{array}\right|.
\]
%% -----File: 006.png---

We then condense these into the column~$\left|\begin{array}{r}
0\\
0\\
\DPtypo{2}{-2}
\end{array}\right|$,
and, supplying from
the second block of the first series the column~$\left|\begin{array}{r}
 3\\
-1\\
-5
\end{array}\right|$, we obtain
$\left|\begin{array}{rr}
 3 &0\\
-1 &0\\
-5 &\DPtypo{2}{-2}
\end{array}\right|$ as the last two columns of the \emph{second} block of the new series;
and proceeding thus we ultimately obtain the two terms $12$,~$12$. Observing
that the $y$-column has the sign~$+$ placed over it, we multiply the first~$12$
by~$+y$, and so form the equation $12y = 12$, which gives $y = 1$. The
values of $z$,~$u$, and~$v$ are similarly found.

% In Dodgson's two examples, the coefficient matrix of the original
% system is at upper left. Each column of the calculation, which Dodgson
% calls a "series", solves the system for one variable. Successive matrices
% in a column are obtained by condensations, reducing the size of the matrix
% until the value of one variable can be read off, with the sign adjusted
% according to the +/- over the corresponding column in the original system.
%
% Next, the solved value of the variable is substituted into the system, the
% last equation/coefficient row is dropped, the constants are "updated", and
% the process begins anew. Only condensations involving the last column need
% to be recomputed, so the second and subsequent series involve only 2 x n
% matrices. In the matrix at the top of each series, the two columns are
% (i) the penultimate column of the preceding series and (ii) the updated
% values of the constants.
It will be seen that when once the given block has been successfully
condensed, and the value of the first unknown obtained, there is no further
danger of the operation being interrupted by the occurrence of ciphers.
\hypertarget{example}
\[
\begin{array}{ccccccc}
 - &+   &-   &+   &-   &+  &   \\
 x &+2y &+ z &- u &+2v &+2 &= 0\\
 x &- y &-2z &- u &- v &-4 &= 0\\
2x &+ y &- z &-2u &- v &-6 &= 0\\
 x &-2y &- z &- u &+2v &+4 &= 0\\
2x &- y &+2z &+ u &-3v &-8 &= 0
\end{array}
\]
\[
\SquashArrays
% First series of blocks
\begin{gathered}
\left|\begin{array}{rrrrrr}
1 & 2 & 1 &-1 & 2 & 2\\
1 &-1 &-2 &-1 &-1 &-4\\
2 & 1 &-1 &-2 &-1 &-6\\
1 &-2 &-1 &-1 & 2 & 4\\
2 &-1 & 2 & 1 &-3 &-8
\end{array}\right|
\\
\left|\begin{array}{rrrrr}
-3 &-3 &-3 & 3 &-6\\
 3 & 3 & 3 &-1 & 2\\
-5 &-3 &-1 &-5 & 8\\
 3 &-5 & 1 & 1 &-4
\end{array}\right|
\\
\left|\begin{array}{rrrr}
  0 & 0 & 6 & 0\\
  6 &-6 & 8 &-2\\
-17 & 8 &-4 & 6
\end{array}\right|
\\
\left|\begin{array}{rrr}
 0 &12 &12\\
18 &40 &-8
\end{array}\right|
\\
\left|\begin{array}{rr} 36 & -72 \end{array}\right| \\
\DPtypo{\because}{\therefore} -36x = -72 \\
x = 2
\end{gathered}
\;
% Second series
\begin{gathered}
\left|\begin{array}{rr}
 2 & 4\\
-1 &-2\\
-1 &-2\\
 2 & 6
\end{array}\right|
\\
\left|\begin{array}{rr}
 3 & 0\\
-1 & 0\\
-5 &-2
\end{array}\right|
\\
\left|\begin{array}{rr}
6 & 0\\
8 &-2
\end{array}\right|
\\
\left|\begin{array}{rr} 12 & 12 \end{array}\right| \\
\therefore 12y = 12 \\
y = 1
\\
\Tall{5}
\end{gathered}
\;
% Third series
\begin{gathered}
\left|\begin{array}{rr}
 2 & 6\\
-1 &-3\\
-1 &-1
\end{array}\right|
\\
\left|\begin{array}{rr}
 3 & 0\\
-1 &-2
\end{array}\right|
\\
\left|\begin{array}{rr} 6 & 6\end{array}\right| \\
\therefore -6z = 6 \\
z = -1
\\
\Tall{5}
\\
\Tall{4}
\end{gathered}
\;
% Fourth series
\begin{gathered}
\left|\begin{array}{rr}
 2 & 5\\
-1 &-1
\end{array}\right|
\\
\left|\begin{array}{rr} 3 & 3 \end{array}\right| \\
\therefore 3u = 3 \\
u = 1
\\
\Tall{5}
\\
\Tall{4}
\\
\Tall{3}
\end{gathered}
\;
% Fifth series
\begin{gathered}
\left|\begin{array}{rr} 2 & 4 \end{array}\right| \\
\therefore -2v = 4 \\
v = -2
\\
\Tall{5}
\\
\Tall{4}
\\
\Tall{3}
\\
\Tall{2}
\end{gathered}
\]

\tb
%
\[
\begin{array}{ccccc}
-  &+   &-   &+   &   \\
5x &+2y &-3z &+ 3 &= 0\\
3x &- y &-2z &+ 7 &= 0\\
2x &+3y &+ z &-12 &= 0
\end{array}
\]
\[
% First series
\begin{gathered}
\left|\begin{array}{rrrr}
5 & 2 &-3 &  3\\
3 &-1 &-2 &  7\\
2 & 3 & 1 &-12
\end{array}\right|
\\
\left|\begin{array}{rrr}
-11 &-7 &-15\\
 11 & 5 & 17
\end{array}\right|
\\
\left|\begin{array}{cc} -22 & 22 \end{array}\right| \\
\therefore 22x =22 \\
x = 1
\end{gathered}
\quad
\begin{gathered}
\left|\begin{array}{rr}
-3 & 8\\
-2 &10
\end{array}\right|
\\
\left|\begin{array}{cc} -7 & -14 \end{array}\right| \\
\therefore -7y = -14 \\
y = 2
\\
\Tall{3}
\end{gathered}
\quad
\begin{gathered}
\left|\begin{array}{rr} -3 & 12 \end{array}\right| \\
\therefore 3z = 12 \\
z = 4
\\
\Tall{3}
\\
\Tall{2}
\end{gathered}
\]

The Society then adjourned over the Whitsuntide Recess to Thursday,
May~31.

\BackMatter
%%%%%%%%%%%%%%%%%%%%%%%%% GUTENBERG LICENSE %%%%%%%%%%%%%%%%%%%%%%%%%%
\newpage
\BookMark{0}{PG License.}
\begin{PGtext}
End of the Project Gutenberg EBook of Condensation of Determinants, Being a
New and Brief Method for Computing their Arithmetical Values, by Lewis Carroll (AKA Charles Lutwidge Dodgson)

*** END OF THIS PROJECT GUTENBERG EBOOK CONDENSATION OF DETERMINANTS ***

***** This file should be named 37354-pdf.pdf or 37354-pdf.zip *****
This and all associated files of various formats will be found in:
        http://www.gutenberg.org/3/7/3/5/37354/

Produced by Carlo Traverso, Anna Hall and the Online
Distributed Proofreading Team at http://www.pgdp.net (This
book was produced from scanned images of public domain
material from the Google Print project.)


Updated editions will replace the previous one--the old editions
will be renamed.

Creating the works from public domain print editions means that no
one owns a United States copyright in these works, so the Foundation
(and you!) can copy and distribute it in the United States without
permission and without paying copyright royalties.  Special rules,
set forth in the General Terms of Use part of this license, apply to
copying and distributing Project Gutenberg-tm electronic works to
protect the PROJECT GUTENBERG-tm concept and trademark.  Project
Gutenberg is a registered trademark, and may not be used if you
charge for the eBooks, unless you receive specific permission.  If you
do not charge anything for copies of this eBook, complying with the
rules is very easy.  You may use this eBook for nearly any purpose
such as creation of derivative works, reports, performances and
research.  They may be modified and printed and given away--you may do
practically ANYTHING with public domain eBooks.  Redistribution is
subject to the trademark license, especially commercial
redistribution.



*** START: FULL LICENSE ***

THE FULL PROJECT GUTENBERG LICENSE
PLEASE READ THIS BEFORE YOU DISTRIBUTE OR USE THIS WORK

To protect the Project Gutenberg-tm mission of promoting the free
distribution of electronic works, by using or distributing this work
(or any other work associated in any way with the phrase "Project
Gutenberg"), you agree to comply with all the terms of the Full Project
Gutenberg-tm License (available with this file or online at
http://gutenberg.org/license).


Section 1.  General Terms of Use and Redistributing Project Gutenberg-tm
electronic works

1.A.  By reading or using any part of this Project Gutenberg-tm
electronic work, you indicate that you have read, understand, agree to
and accept all the terms of this license and intellectual property
(trademark/copyright) agreement.  If you do not agree to abide by all
the terms of this agreement, you must cease using and return or destroy
all copies of Project Gutenberg-tm electronic works in your possession.
If you paid a fee for obtaining a copy of or access to a Project
Gutenberg-tm electronic work and you do not agree to be bound by the
terms of this agreement, you may obtain a refund from the person or
entity to whom you paid the fee as set forth in paragraph 1.E.8.

1.B.  "Project Gutenberg" is a registered trademark.  It may only be
used on or associated in any way with an electronic work by people who
agree to be bound by the terms of this agreement.  There are a few
things that you can do with most Project Gutenberg-tm electronic works
even without complying with the full terms of this agreement.  See
paragraph 1.C below.  There are a lot of things you can do with Project
Gutenberg-tm electronic works if you follow the terms of this agreement
and help preserve free future access to Project Gutenberg-tm electronic
works.  See paragraph 1.E below.

1.C.  The Project Gutenberg Literary Archive Foundation ("the Foundation"
or PGLAF), owns a compilation copyright in the collection of Project
Gutenberg-tm electronic works.  Nearly all the individual works in the
collection are in the public domain in the United States.  If an
individual work is in the public domain in the United States and you are
located in the United States, we do not claim a right to prevent you from
copying, distributing, performing, displaying or creating derivative
works based on the work as long as all references to Project Gutenberg
are removed.  Of course, we hope that you will support the Project
Gutenberg-tm mission of promoting free access to electronic works by
freely sharing Project Gutenberg-tm works in compliance with the terms of
this agreement for keeping the Project Gutenberg-tm name associated with
the work.  You can easily comply with the terms of this agreement by
keeping this work in the same format with its attached full Project
Gutenberg-tm License when you share it without charge with others.

1.D.  The copyright laws of the place where you are located also govern
what you can do with this work.  Copyright laws in most countries are in
a constant state of change.  If you are outside the United States, check
the laws of your country in addition to the terms of this agreement
before downloading, copying, displaying, performing, distributing or
creating derivative works based on this work or any other Project
Gutenberg-tm work.  The Foundation makes no representations concerning
the copyright status of any work in any country outside the United
States.

1.E.  Unless you have removed all references to Project Gutenberg:

1.E.1.  The following sentence, with active links to, or other immediate
access to, the full Project Gutenberg-tm License must appear prominently
whenever any copy of a Project Gutenberg-tm work (any work on which the
phrase "Project Gutenberg" appears, or with which the phrase "Project
Gutenberg" is associated) is accessed, displayed, performed, viewed,
copied or distributed:

This eBook is for the use of anyone anywhere at no cost and with
almost no restrictions whatsoever.  You may copy it, give it away or
re-use it under the terms of the Project Gutenberg License included
with this eBook or online at www.gutenberg.org

1.E.2.  If an individual Project Gutenberg-tm electronic work is derived
from the public domain (does not contain a notice indicating that it is
posted with permission of the copyright holder), the work can be copied
and distributed to anyone in the United States without paying any fees
or charges.  If you are redistributing or providing access to a work
with the phrase "Project Gutenberg" associated with or appearing on the
work, you must comply either with the requirements of paragraphs 1.E.1
through 1.E.7 or obtain permission for the use of the work and the
Project Gutenberg-tm trademark as set forth in paragraphs 1.E.8 or
1.E.9.

1.E.3.  If an individual Project Gutenberg-tm electronic work is posted
with the permission of the copyright holder, your use and distribution
must comply with both paragraphs 1.E.1 through 1.E.7 and any additional
terms imposed by the copyright holder.  Additional terms will be linked
to the Project Gutenberg-tm License for all works posted with the
permission of the copyright holder found at the beginning of this work.

1.E.4.  Do not unlink or detach or remove the full Project Gutenberg-tm
License terms from this work, or any files containing a part of this
work or any other work associated with Project Gutenberg-tm.

1.E.5.  Do not copy, display, perform, distribute or redistribute this
electronic work, or any part of this electronic work, without
prominently displaying the sentence set forth in paragraph 1.E.1 with
active links or immediate access to the full terms of the Project
Gutenberg-tm License.

1.E.6.  You may convert to and distribute this work in any binary,
compressed, marked up, nonproprietary or proprietary form, including any
word processing or hypertext form.  However, if you provide access to or
distribute copies of a Project Gutenberg-tm work in a format other than
"Plain Vanilla ASCII" or other format used in the official version
posted on the official Project Gutenberg-tm web site (www.gutenberg.org),
you must, at no additional cost, fee or expense to the user, provide a
copy, a means of exporting a copy, or a means of obtaining a copy upon
request, of the work in its original "Plain Vanilla ASCII" or other
form.  Any alternate format must include the full Project Gutenberg-tm
License as specified in paragraph 1.E.1.

1.E.7.  Do not charge a fee for access to, viewing, displaying,
performing, copying or distributing any Project Gutenberg-tm works
unless you comply with paragraph 1.E.8 or 1.E.9.

1.E.8.  You may charge a reasonable fee for copies of or providing
access to or distributing Project Gutenberg-tm electronic works provided
that

- You pay a royalty fee of 20% of the gross profits you derive from
     the use of Project Gutenberg-tm works calculated using the method
     you already use to calculate your applicable taxes.  The fee is
     owed to the owner of the Project Gutenberg-tm trademark, but he
     has agreed to donate royalties under this paragraph to the
     Project Gutenberg Literary Archive Foundation.  Royalty payments
     must be paid within 60 days following each date on which you
     prepare (or are legally required to prepare) your periodic tax
     returns.  Royalty payments should be clearly marked as such and
     sent to the Project Gutenberg Literary Archive Foundation at the
     address specified in Section 4, "Information about donations to
     the Project Gutenberg Literary Archive Foundation."

- You provide a full refund of any money paid by a user who notifies
     you in writing (or by e-mail) within 30 days of receipt that s/he
     does not agree to the terms of the full Project Gutenberg-tm
     License.  You must require such a user to return or
     destroy all copies of the works possessed in a physical medium
     and discontinue all use of and all access to other copies of
     Project Gutenberg-tm works.

- You provide, in accordance with paragraph 1.F.3, a full refund of any
     money paid for a work or a replacement copy, if a defect in the
     electronic work is discovered and reported to you within 90 days
     of receipt of the work.

- You comply with all other terms of this agreement for free
     distribution of Project Gutenberg-tm works.

1.E.9.  If you wish to charge a fee or distribute a Project Gutenberg-tm
electronic work or group of works on different terms than are set
forth in this agreement, you must obtain permission in writing from
both the Project Gutenberg Literary Archive Foundation and Michael
Hart, the owner of the Project Gutenberg-tm trademark.  Contact the
Foundation as set forth in Section 3 below.

1.F.

1.F.1.  Project Gutenberg volunteers and employees expend considerable
effort to identify, do copyright research on, transcribe and proofread
public domain works in creating the Project Gutenberg-tm
collection.  Despite these efforts, Project Gutenberg-tm electronic
works, and the medium on which they may be stored, may contain
"Defects," such as, but not limited to, incomplete, inaccurate or
corrupt data, transcription errors, a copyright or other intellectual
property infringement, a defective or damaged disk or other medium, a
computer virus, or computer codes that damage or cannot be read by
your equipment.

1.F.2.  LIMITED WARRANTY, DISCLAIMER OF DAMAGES - Except for the "Right
of Replacement or Refund" described in paragraph 1.F.3, the Project
Gutenberg Literary Archive Foundation, the owner of the Project
Gutenberg-tm trademark, and any other party distributing a Project
Gutenberg-tm electronic work under this agreement, disclaim all
liability to you for damages, costs and expenses, including legal
fees.  YOU AGREE THAT YOU HAVE NO REMEDIES FOR NEGLIGENCE, STRICT
LIABILITY, BREACH OF WARRANTY OR BREACH OF CONTRACT EXCEPT THOSE
PROVIDED IN PARAGRAPH 1.F.3.  YOU AGREE THAT THE FOUNDATION, THE
TRADEMARK OWNER, AND ANY DISTRIBUTOR UNDER THIS AGREEMENT WILL NOT BE
LIABLE TO YOU FOR ACTUAL, DIRECT, INDIRECT, CONSEQUENTIAL, PUNITIVE OR
INCIDENTAL DAMAGES EVEN IF YOU GIVE NOTICE OF THE POSSIBILITY OF SUCH
DAMAGE.

1.F.3.  LIMITED RIGHT OF REPLACEMENT OR REFUND - If you discover a
defect in this electronic work within 90 days of receiving it, you can
receive a refund of the money (if any) you paid for it by sending a
written explanation to the person you received the work from.  If you
received the work on a physical medium, you must return the medium with
your written explanation.  The person or entity that provided you with
the defective work may elect to provide a replacement copy in lieu of a
refund.  If you received the work electronically, the person or entity
providing it to you may choose to give you a second opportunity to
receive the work electronically in lieu of a refund.  If the second copy
is also defective, you may demand a refund in writing without further
opportunities to fix the problem.

1.F.4.  Except for the limited right of replacement or refund set forth
in paragraph 1.F.3, this work is provided to you 'AS-IS' WITH NO OTHER
WARRANTIES OF ANY KIND, EXPRESS OR IMPLIED, INCLUDING BUT NOT LIMITED TO
WARRANTIES OF MERCHANTIBILITY OR FITNESS FOR ANY PURPOSE.

1.F.5.  Some states do not allow disclaimers of certain implied
warranties or the exclusion or limitation of certain types of damages.
If any disclaimer or limitation set forth in this agreement violates the
law of the state applicable to this agreement, the agreement shall be
interpreted to make the maximum disclaimer or limitation permitted by
the applicable state law.  The invalidity or unenforceability of any
provision of this agreement shall not void the remaining provisions.

1.F.6.  INDEMNITY - You agree to indemnify and hold the Foundation, the
trademark owner, any agent or employee of the Foundation, anyone
providing copies of Project Gutenberg-tm electronic works in accordance
with this agreement, and any volunteers associated with the production,
promotion and distribution of Project Gutenberg-tm electronic works,
harmless from all liability, costs and expenses, including legal fees,
that arise directly or indirectly from any of the following which you do
or cause to occur: (a) distribution of this or any Project Gutenberg-tm
work, (b) alteration, modification, or additions or deletions to any
Project Gutenberg-tm work, and (c) any Defect you cause.


Section  2.  Information about the Mission of Project Gutenberg-tm

Project Gutenberg-tm is synonymous with the free distribution of
electronic works in formats readable by the widest variety of computers
including obsolete, old, middle-aged and new computers.  It exists
because of the efforts of hundreds of volunteers and donations from
people in all walks of life.

Volunteers and financial support to provide volunteers with the
assistance they need, are critical to reaching Project Gutenberg-tm's
goals and ensuring that the Project Gutenberg-tm collection will
remain freely available for generations to come.  In 2001, the Project
Gutenberg Literary Archive Foundation was created to provide a secure
and permanent future for Project Gutenberg-tm and future generations.
To learn more about the Project Gutenberg Literary Archive Foundation
and how your efforts and donations can help, see Sections 3 and 4
and the Foundation web page at http://www.pglaf.org.


Section 3.  Information about the Project Gutenberg Literary Archive
Foundation

The Project Gutenberg Literary Archive Foundation is a non profit
501(c)(3) educational corporation organized under the laws of the
state of Mississippi and granted tax exempt status by the Internal
Revenue Service.  The Foundation's EIN or federal tax identification
number is 64-6221541.  Its 501(c)(3) letter is posted at
http://pglaf.org/fundraising.  Contributions to the Project Gutenberg
Literary Archive Foundation are tax deductible to the full extent
permitted by U.S. federal laws and your state's laws.

The Foundation's principal office is located at 4557 Melan Dr. S.
Fairbanks, AK, 99712., but its volunteers and employees are scattered
throughout numerous locations.  Its business office is located at
809 North 1500 West, Salt Lake City, UT 84116, (801) 596-1887, email
business@pglaf.org.  Email contact links and up to date contact
information can be found at the Foundation's web site and official
page at http://pglaf.org

For additional contact information:
     Dr. Gregory B. Newby
     Chief Executive and Director
     gbnewby@pglaf.org


Section 4.  Information about Donations to the Project Gutenberg
Literary Archive Foundation

Project Gutenberg-tm depends upon and cannot survive without wide
spread public support and donations to carry out its mission of
increasing the number of public domain and licensed works that can be
freely distributed in machine readable form accessible by the widest
array of equipment including outdated equipment.  Many small donations
($1 to $5,000) are particularly important to maintaining tax exempt
status with the IRS.

The Foundation is committed to complying with the laws regulating
charities and charitable donations in all 50 states of the United
States.  Compliance requirements are not uniform and it takes a
considerable effort, much paperwork and many fees to meet and keep up
with these requirements.  We do not solicit donations in locations
where we have not received written confirmation of compliance.  To
SEND DONATIONS or determine the status of compliance for any
particular state visit http://pglaf.org

While we cannot and do not solicit contributions from states where we
have not met the solicitation requirements, we know of no prohibition
against accepting unsolicited donations from donors in such states who
approach us with offers to donate.

International donations are gratefully accepted, but we cannot make
any statements concerning tax treatment of donations received from
outside the United States.  U.S. laws alone swamp our small staff.

Please check the Project Gutenberg Web pages for current donation
methods and addresses.  Donations are accepted in a number of other
ways including checks, online payments and credit card donations.
To donate, please visit: http://pglaf.org/donate


Section 5.  General Information About Project Gutenberg-tm electronic
works.

Professor Michael S. Hart is the originator of the Project Gutenberg-tm
concept of a library of electronic works that could be freely shared
with anyone.  For thirty years, he produced and distributed Project
Gutenberg-tm eBooks with only a loose network of volunteer support.


Project Gutenberg-tm eBooks are often created from several printed
editions, all of which are confirmed as Public Domain in the U.S.
unless a copyright notice is included.  Thus, we do not necessarily
keep eBooks in compliance with any particular paper edition.


Most people start at our Web site which has the main PG search facility:

     http://www.gutenberg.org

This Web site includes information about Project Gutenberg-tm,
including how to make donations to the Project Gutenberg Literary
Archive Foundation, how to help produce our new eBooks, and how to
subscribe to our email newsletter to hear about new eBooks.
\end{PGtext}
% %%%%%%%%%%%%%%%%%%%%%%%%%%%%%%%%%%%%%%%%%%%%%%%%%%%%%%%%%%%%%%%%%%%%%%% %
%                                                                         %
% End of the Project Gutenberg EBook of Condensation of Determinants, Being a
% New and Brief Method for Computing their Arithmetical Values, by Lewis Carroll (AKA Charles Lutwidge Dodgson)
%                                                                         %
% *** END OF THIS PROJECT GUTENBERG EBOOK CONDENSATION OF DETERMINANTS ***%
%                                                                         %
% ***** This file should be named 37354-t.tex or 37354-t.zip *****        %
% This and all associated files of various formats will be found in:      %
%         http://www.gutenberg.org/3/7/3/5/37354/                         %
%                                                                         %
% %%%%%%%%%%%%%%%%%%%%%%%%%%%%%%%%%%%%%%%%%%%%%%%%%%%%%%%%%%%%%%%%%%%%%%% %

\end{document}

###
@ControlwordArguments  =  (
['\\DPtypo', 1, 0, '', '' ,1, 1, '', ''],
['\\BookMark', 1, 0, '', '', 1, 0, '', '']
);
@ControlwordReplace = (
['\\tb', '']
);
###
This is pdfTeX, Version 3.1415926-1.40.10 (TeX Live 2009/Debian) (format=pdflatex 2011.9.6)  8 SEP 2011 09:21
entering extended mode
 %&-line parsing enabled.
**37354-t.tex
(./37354-t.tex
LaTeX2e <2009/09/24>
Babel <v3.8l> and hyphenation patterns for english, usenglishmax, dumylang, noh
yphenation, farsi, arabic, croatian, bulgarian, ukrainian, russian, czech, slov
ak, danish, dutch, finnish, french, basque, ngerman, german, german-x-2009-06-1
9, ngerman-x-2009-06-19, ibycus, monogreek, greek, ancientgreek, hungarian, san
skrit, italian, latin, latvian, lithuanian, mongolian2a, mongolian, bokmal, nyn
orsk, romanian, irish, coptic, serbian, turkish, welsh, esperanto, uppersorbian
, estonian, indonesian, interlingua, icelandic, kurmanji, slovenian, polish, po
rtuguese, spanish, galician, catalan, swedish, ukenglish, pinyin, loaded.
(/usr/share/texmf-texlive/tex/latex/base/book.cls
Document Class: book 2007/10/19 v1.4h Standard LaTeX document class
(/usr/share/texmf-texlive/tex/latex/base/bk12.clo
File: bk12.clo 2007/10/19 v1.4h Standard LaTeX file (size option)
)
\c@part=\count79
\c@chapter=\count80
\c@section=\count81
\c@subsection=\count82
\c@subsubsection=\count83
\c@paragraph=\count84
\c@subparagraph=\count85
\c@figure=\count86
\c@table=\count87
\abovecaptionskip=\skip41
\belowcaptionskip=\skip42
\bibindent=\dimen102
) (/usr/share/texmf-texlive/tex/latex/base/inputenc.sty
Package: inputenc 2008/03/30 v1.1d Input encoding file
\inpenc@prehook=\toks14
\inpenc@posthook=\toks15
(/usr/share/texmf-texlive/tex/latex/base/latin1.def
File: latin1.def 2008/03/30 v1.1d Input encoding file
)) (/usr/share/texmf-texlive/tex/latex/amsmath/amsmath.sty
Package: amsmath 2000/07/18 v2.13 AMS math features
\@mathmargin=\skip43
For additional information on amsmath, use the `?' option.
(/usr/share/texmf-texlive/tex/latex/amsmath/amstext.sty
Package: amstext 2000/06/29 v2.01
(/usr/share/texmf-texlive/tex/latex/amsmath/amsgen.sty
File: amsgen.sty 1999/11/30 v2.0
\@emptytoks=\toks16
\ex@=\dimen103
)) (/usr/share/texmf-texlive/tex/latex/amsmath/amsbsy.sty
Package: amsbsy 1999/11/29 v1.2d
\pmbraise@=\dimen104
) (/usr/share/texmf-texlive/tex/latex/amsmath/amsopn.sty
Package: amsopn 1999/12/14 v2.01 operator names
)
\inf@bad=\count88
LaTeX Info: Redefining \frac on input line 211.
\uproot@=\count89
\leftroot@=\count90
LaTeX Info: Redefining \overline on input line 307.
\classnum@=\count91
\DOTSCASE@=\count92
LaTeX Info: Redefining \ldots on input line 379.
LaTeX Info: Redefining \dots on input line 382.
LaTeX Info: Redefining \cdots on input line 467.
\Mathstrutbox@=\box26
\strutbox@=\box27
\big@size=\dimen105
LaTeX Font Info:    Redeclaring font encoding OML on input line 567.
LaTeX Font Info:    Redeclaring font encoding OMS on input line 568.
\macc@depth=\count93
\c@MaxMatrixCols=\count94
\dotsspace@=\muskip10
\c@parentequation=\count95
\dspbrk@lvl=\count96
\tag@help=\toks17
\row@=\count97
\column@=\count98
\maxfields@=\count99
\andhelp@=\toks18
\eqnshift@=\dimen106
\alignsep@=\dimen107
\tagshift@=\dimen108
\tagwidth@=\dimen109
\totwidth@=\dimen110
\lineht@=\dimen111
\@envbody=\toks19
\multlinegap=\skip44
\multlinetaggap=\skip45
\mathdisplay@stack=\toks20
LaTeX Info: Redefining \[ on input line 2666.
LaTeX Info: Redefining \] on input line 2667.
) (/usr/share/texmf-texlive/tex/latex/amsfonts/amssymb.sty
Package: amssymb 2009/06/22 v3.00
(/usr/share/texmf-texlive/tex/latex/amsfonts/amsfonts.sty
Package: amsfonts 2009/06/22 v3.00 Basic AMSFonts support
\symAMSa=\mathgroup4
\symAMSb=\mathgroup5
LaTeX Font Info:    Overwriting math alphabet `\mathfrak' in version `bold'
(Font)                  U/euf/m/n --> U/euf/b/n on input line 96.
)) (/usr/share/texmf-texlive/tex/latex/base/alltt.sty
Package: alltt 1997/06/16 v2.0g defines alltt environment
) (/usr/share/texmf-texlive/tex/latex/tools/array.sty
Package: array 2008/09/09 v2.4c Tabular extension package (FMi)
\col@sep=\dimen112
\extrarowheight=\dimen113
\NC@list=\toks21
\extratabsurround=\skip46
\backup@length=\skip47
) (/usr/share/texmf-texlive/tex/latex/base/ifthen.sty
Package: ifthen 2001/05/26 v1.1c Standard LaTeX ifthen package (DPC)
) (/usr/share/texmf-texlive/tex/latex/fancyhdr/fancyhdr.sty
\fancy@headwidth=\skip48
\f@ncyO@elh=\skip49
\f@ncyO@erh=\skip50
\f@ncyO@olh=\skip51
\f@ncyO@orh=\skip52
\f@ncyO@elf=\skip53
\f@ncyO@erf=\skip54
\f@ncyO@olf=\skip55
\f@ncyO@orf=\skip56
) (/usr/share/texmf-texlive/tex/latex/geometry/geometry.sty
Package: geometry 2008/12/21 v4.2 Page Geometry
(/usr/share/texmf-texlive/tex/latex/graphics/keyval.sty
Package: keyval 1999/03/16 v1.13 key=value parser (DPC)
\KV@toks@=\toks22
) (/usr/share/texmf-texlive/tex/generic/oberdiek/ifpdf.sty
Package: ifpdf 2009/04/10 v2.0 Provides the ifpdf switch (HO)
Package ifpdf Info: pdfTeX in pdf mode detected.
) (/usr/share/texmf-texlive/tex/generic/oberdiek/ifvtex.sty
Package: ifvtex 2008/11/04 v1.4 Switches for detecting VTeX and its modes (HO)
Package ifvtex Info: VTeX not detected.
)
\Gm@cnth=\count100
\Gm@cntv=\count101
\c@Gm@tempcnt=\count102
\Gm@bindingoffset=\dimen114
\Gm@wd@mp=\dimen115
\Gm@odd@mp=\dimen116
\Gm@even@mp=\dimen117
\Gm@dimlist=\toks23
(/usr/share/texmf-texlive/tex/xelatex/xetexconfig/geometry.cfg))

LaTeX Warning: You have requested, on input line 99, version
               `2010/09/12' of package geometry,
               but only version
               `2008/12/21 v4.2 Page Geometry'
               is available.

(/usr/share/texmf-texlive/tex/latex/hyperref/hyperref.sty
Package: hyperref 2009/10/09 v6.79a Hypertext links for LaTeX
(/usr/share/texmf-texlive/tex/generic/ifxetex/ifxetex.sty
Package: ifxetex 2009/01/23 v0.5 Provides ifxetex conditional
) (/usr/share/texmf-texlive/tex/latex/oberdiek/hycolor.sty
Package: hycolor 2009/10/02 v1.5 Code for color options of hyperref/bookmark (H
O)
(/usr/share/texmf-texlive/tex/latex/oberdiek/xcolor-patch.sty
Package: xcolor-patch 2009/10/02 xcolor patch
))
\@linkdim=\dimen118
\Hy@linkcounter=\count103
\Hy@pagecounter=\count104
(/usr/share/texmf-texlive/tex/latex/hyperref/pd1enc.def
File: pd1enc.def 2009/10/09 v6.79a Hyperref: PDFDocEncoding definition (HO)
) (/usr/share/texmf-texlive/tex/generic/oberdiek/etexcmds.sty
Package: etexcmds 2007/12/12 v1.2 Prefix for e-TeX command names (HO)
(/usr/share/texmf-texlive/tex/generic/oberdiek/infwarerr.sty
Package: infwarerr 2007/09/09 v1.2 Providing info/warning/message (HO)
)
Package etexcmds Info: Could not find \expanded.
(etexcmds)             That can mean that you are not using pdfTeX 1.50 or
(etexcmds)             that some package has redefined \expanded.
(etexcmds)             In the latter case, load this package earlier.
) (/etc/texmf/tex/latex/config/hyperref.cfg
File: hyperref.cfg 2002/06/06 v1.2 hyperref configuration of TeXLive
) (/usr/share/texmf-texlive/tex/latex/oberdiek/kvoptions.sty
Package: kvoptions 2009/08/13 v3.4 Keyval support for LaTeX options (HO)
(/usr/share/texmf-texlive/tex/generic/oberdiek/kvsetkeys.sty
Package: kvsetkeys 2009/07/30 v1.5 Key value parser with default handler suppor
t (HO)
))
Package hyperref Info: Option `hyperfootnotes' set `false' on input line 2864.
Package hyperref Info: Option `bookmarks' set `true' on input line 2864.
Package hyperref Info: Option `linktocpage' set `false' on input line 2864.
Package hyperref Info: Option `pdfdisplaydoctitle' set `true' on input line 286
4.
Package hyperref Info: Option `pdfpagelabels' set `true' on input line 2864.
Package hyperref Info: Option `bookmarksopen' set `true' on input line 2864.
Package hyperref Info: Option `colorlinks' set `true' on input line 2864.
Package hyperref Info: Hyper figures OFF on input line 2975.
Package hyperref Info: Link nesting OFF on input line 2980.
Package hyperref Info: Hyper index ON on input line 2983.
Package hyperref Info: Plain pages OFF on input line 2990.
Package hyperref Info: Backreferencing OFF on input line 2995.
Implicit mode ON; LaTeX internals redefined
Package hyperref Info: Bookmarks ON on input line 3191.
(/usr/share/texmf-texlive/tex/latex/ltxmisc/url.sty
\Urlmuskip=\muskip11
Package: url 2006/04/12  ver 3.3  Verb mode for urls, etc.
)
LaTeX Info: Redefining \url on input line 3428.
(/usr/share/texmf-texlive/tex/generic/oberdiek/bitset.sty
Package: bitset 2007/09/28 v1.0 Data type bit set (HO)
(/usr/share/texmf-texlive/tex/generic/oberdiek/intcalc.sty
Package: intcalc 2007/09/27 v1.1 Expandable integer calculations (HO)
) (/usr/share/texmf-texlive/tex/generic/oberdiek/bigintcalc.sty
Package: bigintcalc 2007/11/11 v1.1 Expandable big integer calculations (HO)
(/usr/share/texmf-texlive/tex/generic/oberdiek/pdftexcmds.sty
Package: pdftexcmds 2009/09/23 v0.6 LuaTeX support for pdfTeX utility functions
 (HO)
(/usr/share/texmf-texlive/tex/generic/oberdiek/ifluatex.sty
Package: ifluatex 2009/04/17 v1.2 Provides the ifluatex switch (HO)
Package ifluatex Info: LuaTeX not detected.
) (/usr/share/texmf-texlive/tex/generic/oberdiek/ltxcmds.sty
Package: ltxcmds 2009/08/05 v1.0 Some LaTeX kernel commands for general use (HO
)
)
Package pdftexcmds Info: LuaTeX not detected.
Package pdftexcmds Info: \pdf@primitive is available.
Package pdftexcmds Info: \pdf@ifprimitive is available.
)))
\Fld@menulength=\count105
\Field@Width=\dimen119
\Fld@charsize=\dimen120
\Field@toks=\toks24
Package hyperref Info: Hyper figures OFF on input line 4377.
Package hyperref Info: Link nesting OFF on input line 4382.
Package hyperref Info: Hyper index ON on input line 4385.
Package hyperref Info: backreferencing OFF on input line 4392.
Package hyperref Info: Link coloring ON on input line 4395.
Package hyperref Info: Link coloring with OCG OFF on input line 4402.
Package hyperref Info: PDF/A mode OFF on input line 4407.
(/usr/share/texmf-texlive/tex/generic/oberdiek/atbegshi.sty
Package: atbegshi 2008/07/31 v1.9 At begin shipout hook (HO)
)
\Hy@abspage=\count106
\c@Item=\count107
)
*hyperref using driver hpdftex*
(/usr/share/texmf-texlive/tex/latex/hyperref/hpdftex.def
File: hpdftex.def 2009/10/09 v6.79a Hyperref driver for pdfTeX
\Fld@listcount=\count108
)

LaTeX Warning: You have requested, on input line 119, version
               `2011/04/17' of package hyperref,
               but only version
               `2009/10/09 v6.79a Hypertext links for LaTeX'
               is available.

\c@RowCount=\count109
(./37354-t.aux)
\openout1 = `37354-t.aux'.

LaTeX Font Info:    Checking defaults for OML/cmm/m/it on input line 182.
LaTeX Font Info:    ... okay on input line 182.
LaTeX Font Info:    Checking defaults for T1/cmr/m/n on input line 182.
LaTeX Font Info:    ... okay on input line 182.
LaTeX Font Info:    Checking defaults for OT1/cmr/m/n on input line 182.
LaTeX Font Info:    ... okay on input line 182.
LaTeX Font Info:    Checking defaults for OMS/cmsy/m/n on input line 182.
LaTeX Font Info:    ... okay on input line 182.
LaTeX Font Info:    Checking defaults for OMX/cmex/m/n on input line 182.
LaTeX Font Info:    ... okay on input line 182.
LaTeX Font Info:    Checking defaults for U/cmr/m/n on input line 182.
LaTeX Font Info:    ... okay on input line 182.
LaTeX Font Info:    Checking defaults for PD1/pdf/m/n on input line 182.
LaTeX Font Info:    ... okay on input line 182.
*geometry auto-detecting driver*
*geometry detected driver: pdftex*
-------------------- Geometry parameters
paper: user defined
landscape: --
twocolumn: --
twoside: true
asymmetric: --
h-parts: 9.03375pt, 361.34999pt, 9.03375pt
v-parts: 13.98709pt, 543.19225pt, 20.98065pt
hmarginratio: 1:1
vmarginratio: 2:3
lines: --
heightrounded: --
bindingoffset: 0.0pt
truedimen: --
includehead: true
includefoot: true
includemp: --
driver: pdftex
-------------------- Page layout dimensions and switches
\paperwidth  379.4175pt
\paperheight 578.15999pt
\textwidth  361.34999pt
\textheight 481.31845pt
\oddsidemargin  -63.23624pt
\evensidemargin -63.23624pt
\topmargin  -58.2829pt
\headheight 12.0pt
\headsep    19.8738pt
\footskip   30.0pt
\marginparwidth 98.0pt
\marginparsep   7.0pt
\columnsep  10.0pt
\skip\footins  10.8pt plus 4.0pt minus 2.0pt
\hoffset 0.0pt
\voffset 0.0pt
\mag 1000
\@twosidetrue \@mparswitchtrue 
(1in=72.27pt, 1cm=28.45pt)
-----------------------
(/usr/share/texmf-texlive/tex/latex/graphics/color.sty
Package: color 2005/11/14 v1.0j Standard LaTeX Color (DPC)
(/etc/texmf/tex/latex/config/color.cfg
File: color.cfg 2007/01/18 v1.5 color configuration of teTeX/TeXLive
)
Package color Info: Driver file: pdftex.def on input line 130.
(/usr/share/texmf-texlive/tex/latex/pdftex-def/pdftex.def
File: pdftex.def 2009/08/25 v0.04m Graphics/color for pdfTeX
\Gread@gobject=\count110
(/usr/share/texmf/tex/context/base/supp-pdf.mkii
[Loading MPS to PDF converter (version 2006.09.02).]
\scratchcounter=\count111
\scratchdimen=\dimen121
\scratchbox=\box28
\nofMPsegments=\count112
\nofMParguments=\count113
\everyMPshowfont=\toks25
\MPscratchCnt=\count114
\MPscratchDim=\dimen122
\MPnumerator=\count115
\everyMPtoPDFconversion=\toks26
)))
Package hyperref Info: Link coloring ON on input line 182.
(/usr/share/texmf-texlive/tex/latex/hyperref/nameref.sty
Package: nameref 2007/05/29 v2.31 Cross-referencing by name of section
(/usr/share/texmf-texlive/tex/latex/oberdiek/refcount.sty
Package: refcount 2008/08/11 v3.1 Data extraction from references (HO)
)
\c@section@level=\count116
)
LaTeX Info: Redefining \ref on input line 182.
LaTeX Info: Redefining \pageref on input line 182.
(./37354-t.out) (./37354-t.out)
\@outlinefile=\write3
\openout3 = `37354-t.out'.

\AtBeginShipoutBox=\box29

Overfull \hbox (84.90642pt too wide) in paragraph at lines 187--187
[]\OT1/cmtt/m/n/8 and Brief Method for Computing their Arithmetical Values, by 
Lewis Carroll (AKA Charles Lutwidge Dodgson)[] 
 []


Overfull \hbox (84.90642pt too wide) in paragraph at lines 195--195
[]\OT1/cmtt/m/n/8 Title: Condensation of Determinants, Being a New and Brief Me
thod for Computing their Arithmetical Values[] 
 []

[1

{/var/lib/texmf/fonts/map/pdftex/updmap/pdftex.map}] [2

] [1



]
LaTeX Font Info:    Try loading font information for U+msa on input line 254.
(/usr/share/texmf-texlive/tex/latex/amsfonts/umsa.fd
File: umsa.fd 2009/06/22 v3.00 AMS symbols A
)
LaTeX Font Info:    Try loading font information for U+msb on input line 254.
(/usr/share/texmf-texlive/tex/latex/amsfonts/umsb.fd
File: umsb.fd 2009/06/22 v3.00 AMS symbols B
) [1



] [2] [3] [4] [5] [6] [7] [8] [9] [10]
Overfull \hbox (101.90666pt too wide) in paragraph at lines 886--886
[]\OT1/cmtt/m/n/8 New and Brief Method for Computing their Arithmetical Values,
 by Lewis Carroll (AKA Charles Lutwidge Dodgson)[] 
 []

[11



] [12] [13] [14] [15] [16] [17] (./37354-t.aux)

 *File List*
    book.cls    2007/10/19 v1.4h Standard LaTeX document class
    bk12.clo    2007/10/19 v1.4h Standard LaTeX file (size option)
inputenc.sty    2008/03/30 v1.1d Input encoding file
  latin1.def    2008/03/30 v1.1d Input encoding file
 amsmath.sty    2000/07/18 v2.13 AMS math features
 amstext.sty    2000/06/29 v2.01
  amsgen.sty    1999/11/30 v2.0
  amsbsy.sty    1999/11/29 v1.2d
  amsopn.sty    1999/12/14 v2.01 operator names
 amssymb.sty    2009/06/22 v3.00
amsfonts.sty    2009/06/22 v3.00 Basic AMSFonts support
   alltt.sty    1997/06/16 v2.0g defines alltt environment
   array.sty    2008/09/09 v2.4c Tabular extension package (FMi)
  ifthen.sty    2001/05/26 v1.1c Standard LaTeX ifthen package (DPC)
fancyhdr.sty    
geometry.sty    2008/12/21 v4.2 Page Geometry
  keyval.sty    1999/03/16 v1.13 key=value parser (DPC)
   ifpdf.sty    2009/04/10 v2.0 Provides the ifpdf switch (HO)
  ifvtex.sty    2008/11/04 v1.4 Switches for detecting VTeX and its modes (HO)
geometry.cfg
hyperref.sty    2009/10/09 v6.79a Hypertext links for LaTeX
 ifxetex.sty    2009/01/23 v0.5 Provides ifxetex conditional
 hycolor.sty    2009/10/02 v1.5 Code for color options of hyperref/bookmark (HO
)
xcolor-patch.sty    2009/10/02 xcolor patch
  pd1enc.def    2009/10/09 v6.79a Hyperref: PDFDocEncoding definition (HO)
etexcmds.sty    2007/12/12 v1.2 Prefix for e-TeX command names (HO)
infwarerr.sty    2007/09/09 v1.2 Providing info/warning/message (HO)
hyperref.cfg    2002/06/06 v1.2 hyperref configuration of TeXLive
kvoptions.sty    2009/08/13 v3.4 Keyval support for LaTeX options (HO)
kvsetkeys.sty    2009/07/30 v1.5 Key value parser with default handler support 
(HO)
     url.sty    2006/04/12  ver 3.3  Verb mode for urls, etc.
  bitset.sty    2007/09/28 v1.0 Data type bit set (HO)
 intcalc.sty    2007/09/27 v1.1 Expandable integer calculations (HO)
bigintcalc.sty    2007/11/11 v1.1 Expandable big integer calculations (HO)
pdftexcmds.sty    2009/09/23 v0.6 LuaTeX support for pdfTeX utility functions (
HO)
ifluatex.sty    2009/04/17 v1.2 Provides the ifluatex switch (HO)
 ltxcmds.sty    2009/08/05 v1.0 Some LaTeX kernel commands for general use (HO)

atbegshi.sty    2008/07/31 v1.9 At begin shipout hook (HO)
 hpdftex.def    2009/10/09 v6.79a Hyperref driver for pdfTeX
   color.sty    2005/11/14 v1.0j Standard LaTeX Color (DPC)
   color.cfg    2007/01/18 v1.5 color configuration of teTeX/TeXLive
  pdftex.def    2009/08/25 v0.04m Graphics/color for pdfTeX
supp-pdf.mkii
 nameref.sty    2007/05/29 v2.31 Cross-referencing by name of section
refcount.sty    2008/08/11 v3.1 Data extraction from references (HO)
 37354-t.out
 37354-t.out
    umsa.fd    2009/06/22 v3.00 AMS symbols A
    umsb.fd    2009/06/22 v3.00 AMS symbols B
 ***********

 ) 
Here is how much of TeX's memory you used:
 5540 strings out of 493848
 77063 string characters out of 1152824
 174243 words of memory out of 3000000
 8700 multiletter control sequences out of 15000+50000
 10943 words of font info for 42 fonts, out of 3000000 for 9000
 714 hyphenation exceptions out of 8191
 37i,20n,43p,298b,458s stack positions out of 5000i,500n,10000p,200000b,50000s
</usr/share/texmf-texlive/fonts/type1/public/amsfonts/cm/cmcsc10.pfb></usr/sh
are/texmf-texlive/fonts/type1/public/amsfonts/cm/cmex10.pfb></usr/share/texmf-t
exlive/fonts/type1/public/amsfonts/cm/cmmi12.pfb></usr/share/texmf-texlive/font
s/type1/public/amsfonts/cm/cmmi8.pfb></usr/share/texmf-texlive/fonts/type1/publ
ic/amsfonts/cm/cmr10.pfb></usr/share/texmf-texlive/fonts/type1/public/amsfonts/
cm/cmr12.pfb></usr/share/texmf-texlive/fonts/type1/public/amsfonts/cm/cmr17.pfb
></usr/share/texmf-texlive/fonts/type1/public/amsfonts/cm/cmr8.pfb></usr/share/
texmf-texlive/fonts/type1/public/amsfonts/cm/cmsy10.pfb></usr/share/texmf-texli
ve/fonts/type1/public/amsfonts/cm/cmsy8.pfb></usr/share/texmf-texlive/fonts/typ
e1/public/amsfonts/cm/cmti12.pfb></usr/share/texmf-texlive/fonts/type1/public/a
msfonts/cm/cmtt8.pfb></usr/share/texmf-texlive/fonts/type1/public/amsfonts/symb
ols/msam10.pfb>
Output written on 37354-t.pdf (20 pages, 194441 bytes).
PDF statistics:
 173 PDF objects out of 1000 (max. 8388607)
 31 named destinations out of 1000 (max. 500000)
 33 words of extra memory for PDF output out of 10000 (max. 10000000)

